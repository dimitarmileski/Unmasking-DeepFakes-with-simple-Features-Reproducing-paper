\documentclass[11pt]{article}

    \usepackage[breakable]{tcolorbox}
    \usepackage{parskip} % Stop auto-indenting (to mimic markdown behaviour)
    
    \usepackage{iftex}
    \ifPDFTeX
    	\usepackage[T1]{fontenc}
    	\usepackage{mathpazo}
    \else
    	\usepackage{fontspec}
    \fi

    % Basic figure setup, for now with no caption control since it's done
    % automatically by Pandoc (which extracts ![](path) syntax from Markdown).
    \usepackage{graphicx}
    % Maintain compatibility with old templates. Remove in nbconvert 6.0
    \let\Oldincludegraphics\includegraphics
    % Ensure that by default, figures have no caption (until we provide a
    % proper Figure object with a Caption API and a way to capture that
    % in the conversion process - todo).
    \usepackage{caption}
    \DeclareCaptionFormat{nocaption}{}
    \captionsetup{format=nocaption,aboveskip=0pt,belowskip=0pt}

    \usepackage[Export]{adjustbox} % Used to constrain images to a maximum size
    \adjustboxset{max size={0.9\linewidth}{0.9\paperheight}}
    \usepackage{float}
    \floatplacement{figure}{H} % forces figures to be placed at the correct location
    \usepackage{xcolor} % Allow colors to be defined
    \usepackage{enumerate} % Needed for markdown enumerations to work
    \usepackage{geometry} % Used to adjust the document margins
    \usepackage{amsmath} % Equations
    \usepackage{amssymb} % Equations
    \usepackage{textcomp} % defines textquotesingle
    % Hack from http://tex.stackexchange.com/a/47451/13684:
    \AtBeginDocument{%
        \def\PYZsq{\textquotesingle}% Upright quotes in Pygmentized code
    }
    \usepackage{upquote} % Upright quotes for verbatim code
    \usepackage{eurosym} % defines \euro
    \usepackage[mathletters]{ucs} % Extended unicode (utf-8) support
    \usepackage{fancyvrb} % verbatim replacement that allows latex
    \usepackage{grffile} % extends the file name processing of package graphics 
                         % to support a larger range
    \makeatletter % fix for grffile with XeLaTeX
    \def\Gread@@xetex#1{%
      \IfFileExists{"\Gin@base".bb}%
      {\Gread@eps{\Gin@base.bb}}%
      {\Gread@@xetex@aux#1}%
    }
    \makeatother

    % The hyperref package gives us a pdf with properly built
    % internal navigation ('pdf bookmarks' for the table of contents,
    % internal cross-reference links, web links for URLs, etc.)
    \usepackage{hyperref}
    % The default LaTeX title has an obnoxious amount of whitespace. By default,
    % titling removes some of it. It also provides customization options.
    \usepackage{titling}
    \usepackage{longtable} % longtable support required by pandoc >1.10
    \usepackage{booktabs}  % table support for pandoc > 1.12.2
    \usepackage[inline]{enumitem} % IRkernel/repr support (it uses the enumerate* environment)
    \usepackage[normalem]{ulem} % ulem is needed to support strikethroughs (\sout)
                                % normalem makes italics be italics, not underlines
    \usepackage{mathrsfs}
    

    
    % Colors for the hyperref package
    \definecolor{urlcolor}{rgb}{0,.145,.698}
    \definecolor{linkcolor}{rgb}{.71,0.21,0.01}
    \definecolor{citecolor}{rgb}{.12,.54,.11}

    % ANSI colors
    \definecolor{ansi-black}{HTML}{3E424D}
    \definecolor{ansi-black-intense}{HTML}{282C36}
    \definecolor{ansi-red}{HTML}{E75C58}
    \definecolor{ansi-red-intense}{HTML}{B22B31}
    \definecolor{ansi-green}{HTML}{00A250}
    \definecolor{ansi-green-intense}{HTML}{007427}
    \definecolor{ansi-yellow}{HTML}{DDB62B}
    \definecolor{ansi-yellow-intense}{HTML}{B27D12}
    \definecolor{ansi-blue}{HTML}{208FFB}
    \definecolor{ansi-blue-intense}{HTML}{0065CA}
    \definecolor{ansi-magenta}{HTML}{D160C4}
    \definecolor{ansi-magenta-intense}{HTML}{A03196}
    \definecolor{ansi-cyan}{HTML}{60C6C8}
    \definecolor{ansi-cyan-intense}{HTML}{258F8F}
    \definecolor{ansi-white}{HTML}{C5C1B4}
    \definecolor{ansi-white-intense}{HTML}{A1A6B2}
    \definecolor{ansi-default-inverse-fg}{HTML}{FFFFFF}
    \definecolor{ansi-default-inverse-bg}{HTML}{000000}

    % commands and environments needed by pandoc snippets
    % extracted from the output of `pandoc -s`
    \providecommand{\tightlist}{%
      \setlength{\itemsep}{0pt}\setlength{\parskip}{0pt}}
    \DefineVerbatimEnvironment{Highlighting}{Verbatim}{commandchars=\\\{\}}
    % Add ',fontsize=\small' for more characters per line
    \newenvironment{Shaded}{}{}
    \newcommand{\KeywordTok}[1]{\textcolor[rgb]{0.00,0.44,0.13}{\textbf{{#1}}}}
    \newcommand{\DataTypeTok}[1]{\textcolor[rgb]{0.56,0.13,0.00}{{#1}}}
    \newcommand{\DecValTok}[1]{\textcolor[rgb]{0.25,0.63,0.44}{{#1}}}
    \newcommand{\BaseNTok}[1]{\textcolor[rgb]{0.25,0.63,0.44}{{#1}}}
    \newcommand{\FloatTok}[1]{\textcolor[rgb]{0.25,0.63,0.44}{{#1}}}
    \newcommand{\CharTok}[1]{\textcolor[rgb]{0.25,0.44,0.63}{{#1}}}
    \newcommand{\StringTok}[1]{\textcolor[rgb]{0.25,0.44,0.63}{{#1}}}
    \newcommand{\CommentTok}[1]{\textcolor[rgb]{0.38,0.63,0.69}{\textit{{#1}}}}
    \newcommand{\OtherTok}[1]{\textcolor[rgb]{0.00,0.44,0.13}{{#1}}}
    \newcommand{\AlertTok}[1]{\textcolor[rgb]{1.00,0.00,0.00}{\textbf{{#1}}}}
    \newcommand{\FunctionTok}[1]{\textcolor[rgb]{0.02,0.16,0.49}{{#1}}}
    \newcommand{\RegionMarkerTok}[1]{{#1}}
    \newcommand{\ErrorTok}[1]{\textcolor[rgb]{1.00,0.00,0.00}{\textbf{{#1}}}}
    \newcommand{\NormalTok}[1]{{#1}}
    
    % Additional commands for more recent versions of Pandoc
    \newcommand{\ConstantTok}[1]{\textcolor[rgb]{0.53,0.00,0.00}{{#1}}}
    \newcommand{\SpecialCharTok}[1]{\textcolor[rgb]{0.25,0.44,0.63}{{#1}}}
    \newcommand{\VerbatimStringTok}[1]{\textcolor[rgb]{0.25,0.44,0.63}{{#1}}}
    \newcommand{\SpecialStringTok}[1]{\textcolor[rgb]{0.73,0.40,0.53}{{#1}}}
    \newcommand{\ImportTok}[1]{{#1}}
    \newcommand{\DocumentationTok}[1]{\textcolor[rgb]{0.73,0.13,0.13}{\textit{{#1}}}}
    \newcommand{\AnnotationTok}[1]{\textcolor[rgb]{0.38,0.63,0.69}{\textbf{\textit{{#1}}}}}
    \newcommand{\CommentVarTok}[1]{\textcolor[rgb]{0.38,0.63,0.69}{\textbf{\textit{{#1}}}}}
    \newcommand{\VariableTok}[1]{\textcolor[rgb]{0.10,0.09,0.49}{{#1}}}
    \newcommand{\ControlFlowTok}[1]{\textcolor[rgb]{0.00,0.44,0.13}{\textbf{{#1}}}}
    \newcommand{\OperatorTok}[1]{\textcolor[rgb]{0.40,0.40,0.40}{{#1}}}
    \newcommand{\BuiltInTok}[1]{{#1}}
    \newcommand{\ExtensionTok}[1]{{#1}}
    \newcommand{\PreprocessorTok}[1]{\textcolor[rgb]{0.74,0.48,0.00}{{#1}}}
    \newcommand{\AttributeTok}[1]{\textcolor[rgb]{0.49,0.56,0.16}{{#1}}}
    \newcommand{\InformationTok}[1]{\textcolor[rgb]{0.38,0.63,0.69}{\textbf{\textit{{#1}}}}}
    \newcommand{\WarningTok}[1]{\textcolor[rgb]{0.38,0.63,0.69}{\textbf{\textit{{#1}}}}}
    
    
    % Define a nice break command that doesn't care if a line doesn't already
    % exist.
    \def\br{\hspace*{\fill} \\* }
    % Math Jax compatibility definitions
    \def\gt{>}
    \def\lt{<}
    \let\Oldtex\TeX
    \let\Oldlatex\LaTeX
    \renewcommand{\TeX}{\textrm{\Oldtex}}
    \renewcommand{\LaTeX}{\textrm{\Oldlatex}}
    % Document parameters
    % Document title
    
\title{Mi-IKT First Partail Exam}

    
    
\author{Dimitar Mileski}

    
% Pygments definitions
\makeatletter
\def\PY@reset{\let\PY@it=\relax \let\PY@bf=\relax%
    \let\PY@ul=\relax \let\PY@tc=\relax%
    \let\PY@bc=\relax \let\PY@ff=\relax}
\def\PY@tok#1{\csname PY@tok@#1\endcsname}
\def\PY@toks#1+{\ifx\relax#1\empty\else%
    \PY@tok{#1}\expandafter\PY@toks\fi}
\def\PY@do#1{\PY@bc{\PY@tc{\PY@ul{%
    \PY@it{\PY@bf{\PY@ff{#1}}}}}}}
\def\PY#1#2{\PY@reset\PY@toks#1+\relax+\PY@do{#2}}

\expandafter\def\csname PY@tok@w\endcsname{\def\PY@tc##1{\textcolor[rgb]{0.73,0.73,0.73}{##1}}}
\expandafter\def\csname PY@tok@c\endcsname{\let\PY@it=\textit\def\PY@tc##1{\textcolor[rgb]{0.25,0.50,0.50}{##1}}}
\expandafter\def\csname PY@tok@cp\endcsname{\def\PY@tc##1{\textcolor[rgb]{0.74,0.48,0.00}{##1}}}
\expandafter\def\csname PY@tok@k\endcsname{\let\PY@bf=\textbf\def\PY@tc##1{\textcolor[rgb]{0.00,0.50,0.00}{##1}}}
\expandafter\def\csname PY@tok@kp\endcsname{\def\PY@tc##1{\textcolor[rgb]{0.00,0.50,0.00}{##1}}}
\expandafter\def\csname PY@tok@kt\endcsname{\def\PY@tc##1{\textcolor[rgb]{0.69,0.00,0.25}{##1}}}
\expandafter\def\csname PY@tok@o\endcsname{\def\PY@tc##1{\textcolor[rgb]{0.40,0.40,0.40}{##1}}}
\expandafter\def\csname PY@tok@ow\endcsname{\let\PY@bf=\textbf\def\PY@tc##1{\textcolor[rgb]{0.67,0.13,1.00}{##1}}}
\expandafter\def\csname PY@tok@nb\endcsname{\def\PY@tc##1{\textcolor[rgb]{0.00,0.50,0.00}{##1}}}
\expandafter\def\csname PY@tok@nf\endcsname{\def\PY@tc##1{\textcolor[rgb]{0.00,0.00,1.00}{##1}}}
\expandafter\def\csname PY@tok@nc\endcsname{\let\PY@bf=\textbf\def\PY@tc##1{\textcolor[rgb]{0.00,0.00,1.00}{##1}}}
\expandafter\def\csname PY@tok@nn\endcsname{\let\PY@bf=\textbf\def\PY@tc##1{\textcolor[rgb]{0.00,0.00,1.00}{##1}}}
\expandafter\def\csname PY@tok@ne\endcsname{\let\PY@bf=\textbf\def\PY@tc##1{\textcolor[rgb]{0.82,0.25,0.23}{##1}}}
\expandafter\def\csname PY@tok@nv\endcsname{\def\PY@tc##1{\textcolor[rgb]{0.10,0.09,0.49}{##1}}}
\expandafter\def\csname PY@tok@no\endcsname{\def\PY@tc##1{\textcolor[rgb]{0.53,0.00,0.00}{##1}}}
\expandafter\def\csname PY@tok@nl\endcsname{\def\PY@tc##1{\textcolor[rgb]{0.63,0.63,0.00}{##1}}}
\expandafter\def\csname PY@tok@ni\endcsname{\let\PY@bf=\textbf\def\PY@tc##1{\textcolor[rgb]{0.60,0.60,0.60}{##1}}}
\expandafter\def\csname PY@tok@na\endcsname{\def\PY@tc##1{\textcolor[rgb]{0.49,0.56,0.16}{##1}}}
\expandafter\def\csname PY@tok@nt\endcsname{\let\PY@bf=\textbf\def\PY@tc##1{\textcolor[rgb]{0.00,0.50,0.00}{##1}}}
\expandafter\def\csname PY@tok@nd\endcsname{\def\PY@tc##1{\textcolor[rgb]{0.67,0.13,1.00}{##1}}}
\expandafter\def\csname PY@tok@s\endcsname{\def\PY@tc##1{\textcolor[rgb]{0.73,0.13,0.13}{##1}}}
\expandafter\def\csname PY@tok@sd\endcsname{\let\PY@it=\textit\def\PY@tc##1{\textcolor[rgb]{0.73,0.13,0.13}{##1}}}
\expandafter\def\csname PY@tok@si\endcsname{\let\PY@bf=\textbf\def\PY@tc##1{\textcolor[rgb]{0.73,0.40,0.53}{##1}}}
\expandafter\def\csname PY@tok@se\endcsname{\let\PY@bf=\textbf\def\PY@tc##1{\textcolor[rgb]{0.73,0.40,0.13}{##1}}}
\expandafter\def\csname PY@tok@sr\endcsname{\def\PY@tc##1{\textcolor[rgb]{0.73,0.40,0.53}{##1}}}
\expandafter\def\csname PY@tok@ss\endcsname{\def\PY@tc##1{\textcolor[rgb]{0.10,0.09,0.49}{##1}}}
\expandafter\def\csname PY@tok@sx\endcsname{\def\PY@tc##1{\textcolor[rgb]{0.00,0.50,0.00}{##1}}}
\expandafter\def\csname PY@tok@m\endcsname{\def\PY@tc##1{\textcolor[rgb]{0.40,0.40,0.40}{##1}}}
\expandafter\def\csname PY@tok@gh\endcsname{\let\PY@bf=\textbf\def\PY@tc##1{\textcolor[rgb]{0.00,0.00,0.50}{##1}}}
\expandafter\def\csname PY@tok@gu\endcsname{\let\PY@bf=\textbf\def\PY@tc##1{\textcolor[rgb]{0.50,0.00,0.50}{##1}}}
\expandafter\def\csname PY@tok@gd\endcsname{\def\PY@tc##1{\textcolor[rgb]{0.63,0.00,0.00}{##1}}}
\expandafter\def\csname PY@tok@gi\endcsname{\def\PY@tc##1{\textcolor[rgb]{0.00,0.63,0.00}{##1}}}
\expandafter\def\csname PY@tok@gr\endcsname{\def\PY@tc##1{\textcolor[rgb]{1.00,0.00,0.00}{##1}}}
\expandafter\def\csname PY@tok@ge\endcsname{\let\PY@it=\textit}
\expandafter\def\csname PY@tok@gs\endcsname{\let\PY@bf=\textbf}
\expandafter\def\csname PY@tok@gp\endcsname{\let\PY@bf=\textbf\def\PY@tc##1{\textcolor[rgb]{0.00,0.00,0.50}{##1}}}
\expandafter\def\csname PY@tok@go\endcsname{\def\PY@tc##1{\textcolor[rgb]{0.53,0.53,0.53}{##1}}}
\expandafter\def\csname PY@tok@gt\endcsname{\def\PY@tc##1{\textcolor[rgb]{0.00,0.27,0.87}{##1}}}
\expandafter\def\csname PY@tok@err\endcsname{\def\PY@bc##1{\setlength{\fboxsep}{0pt}\fcolorbox[rgb]{1.00,0.00,0.00}{1,1,1}{\strut ##1}}}
\expandafter\def\csname PY@tok@kc\endcsname{\let\PY@bf=\textbf\def\PY@tc##1{\textcolor[rgb]{0.00,0.50,0.00}{##1}}}
\expandafter\def\csname PY@tok@kd\endcsname{\let\PY@bf=\textbf\def\PY@tc##1{\textcolor[rgb]{0.00,0.50,0.00}{##1}}}
\expandafter\def\csname PY@tok@kn\endcsname{\let\PY@bf=\textbf\def\PY@tc##1{\textcolor[rgb]{0.00,0.50,0.00}{##1}}}
\expandafter\def\csname PY@tok@kr\endcsname{\let\PY@bf=\textbf\def\PY@tc##1{\textcolor[rgb]{0.00,0.50,0.00}{##1}}}
\expandafter\def\csname PY@tok@bp\endcsname{\def\PY@tc##1{\textcolor[rgb]{0.00,0.50,0.00}{##1}}}
\expandafter\def\csname PY@tok@fm\endcsname{\def\PY@tc##1{\textcolor[rgb]{0.00,0.00,1.00}{##1}}}
\expandafter\def\csname PY@tok@vc\endcsname{\def\PY@tc##1{\textcolor[rgb]{0.10,0.09,0.49}{##1}}}
\expandafter\def\csname PY@tok@vg\endcsname{\def\PY@tc##1{\textcolor[rgb]{0.10,0.09,0.49}{##1}}}
\expandafter\def\csname PY@tok@vi\endcsname{\def\PY@tc##1{\textcolor[rgb]{0.10,0.09,0.49}{##1}}}
\expandafter\def\csname PY@tok@vm\endcsname{\def\PY@tc##1{\textcolor[rgb]{0.10,0.09,0.49}{##1}}}
\expandafter\def\csname PY@tok@sa\endcsname{\def\PY@tc##1{\textcolor[rgb]{0.73,0.13,0.13}{##1}}}
\expandafter\def\csname PY@tok@sb\endcsname{\def\PY@tc##1{\textcolor[rgb]{0.73,0.13,0.13}{##1}}}
\expandafter\def\csname PY@tok@sc\endcsname{\def\PY@tc##1{\textcolor[rgb]{0.73,0.13,0.13}{##1}}}
\expandafter\def\csname PY@tok@dl\endcsname{\def\PY@tc##1{\textcolor[rgb]{0.73,0.13,0.13}{##1}}}
\expandafter\def\csname PY@tok@s2\endcsname{\def\PY@tc##1{\textcolor[rgb]{0.73,0.13,0.13}{##1}}}
\expandafter\def\csname PY@tok@sh\endcsname{\def\PY@tc##1{\textcolor[rgb]{0.73,0.13,0.13}{##1}}}
\expandafter\def\csname PY@tok@s1\endcsname{\def\PY@tc##1{\textcolor[rgb]{0.73,0.13,0.13}{##1}}}
\expandafter\def\csname PY@tok@mb\endcsname{\def\PY@tc##1{\textcolor[rgb]{0.40,0.40,0.40}{##1}}}
\expandafter\def\csname PY@tok@mf\endcsname{\def\PY@tc##1{\textcolor[rgb]{0.40,0.40,0.40}{##1}}}
\expandafter\def\csname PY@tok@mh\endcsname{\def\PY@tc##1{\textcolor[rgb]{0.40,0.40,0.40}{##1}}}
\expandafter\def\csname PY@tok@mi\endcsname{\def\PY@tc##1{\textcolor[rgb]{0.40,0.40,0.40}{##1}}}
\expandafter\def\csname PY@tok@il\endcsname{\def\PY@tc##1{\textcolor[rgb]{0.40,0.40,0.40}{##1}}}
\expandafter\def\csname PY@tok@mo\endcsname{\def\PY@tc##1{\textcolor[rgb]{0.40,0.40,0.40}{##1}}}
\expandafter\def\csname PY@tok@ch\endcsname{\let\PY@it=\textit\def\PY@tc##1{\textcolor[rgb]{0.25,0.50,0.50}{##1}}}
\expandafter\def\csname PY@tok@cm\endcsname{\let\PY@it=\textit\def\PY@tc##1{\textcolor[rgb]{0.25,0.50,0.50}{##1}}}
\expandafter\def\csname PY@tok@cpf\endcsname{\let\PY@it=\textit\def\PY@tc##1{\textcolor[rgb]{0.25,0.50,0.50}{##1}}}
\expandafter\def\csname PY@tok@c1\endcsname{\let\PY@it=\textit\def\PY@tc##1{\textcolor[rgb]{0.25,0.50,0.50}{##1}}}
\expandafter\def\csname PY@tok@cs\endcsname{\let\PY@it=\textit\def\PY@tc##1{\textcolor[rgb]{0.25,0.50,0.50}{##1}}}

\def\PYZbs{\char`\\}
\def\PYZus{\char`\_}
\def\PYZob{\char`\{}
\def\PYZcb{\char`\}}
\def\PYZca{\char`\^}
\def\PYZam{\char`\&}
\def\PYZlt{\char`\<}
\def\PYZgt{\char`\>}
\def\PYZsh{\char`\#}
\def\PYZpc{\char`\%}
\def\PYZdl{\char`\$}
\def\PYZhy{\char`\-}
\def\PYZsq{\char`\'}
\def\PYZdq{\char`\"}
\def\PYZti{\char`\~}
% for compatibility with earlier versions
\def\PYZat{@}
\def\PYZlb{[}
\def\PYZrb{]}
\makeatother


    % For linebreaks inside Verbatim environment from package fancyvrb. 
    \makeatletter
        \newbox\Wrappedcontinuationbox 
        \newbox\Wrappedvisiblespacebox 
        \newcommand*\Wrappedvisiblespace {\textcolor{red}{\textvisiblespace}} 
        \newcommand*\Wrappedcontinuationsymbol {\textcolor{red}{\llap{\tiny$\m@th\hookrightarrow$}}} 
        \newcommand*\Wrappedcontinuationindent {3ex } 
        \newcommand*\Wrappedafterbreak {\kern\Wrappedcontinuationindent\copy\Wrappedcontinuationbox} 
        % Take advantage of the already applied Pygments mark-up to insert 
        % potential linebreaks for TeX processing. 
        %        {, <, #, %, $, ' and ": go to next line. 
        %        _, }, ^, &, >, - and ~: stay at end of broken line. 
        % Use of \textquotesingle for straight quote. 
        \newcommand*\Wrappedbreaksatspecials {% 
            \def\PYGZus{\discretionary{\char`\_}{\Wrappedafterbreak}{\char`\_}}% 
            \def\PYGZob{\discretionary{}{\Wrappedafterbreak\char`\{}{\char`\{}}% 
            \def\PYGZcb{\discretionary{\char`\}}{\Wrappedafterbreak}{\char`\}}}% 
            \def\PYGZca{\discretionary{\char`\^}{\Wrappedafterbreak}{\char`\^}}% 
            \def\PYGZam{\discretionary{\char`\&}{\Wrappedafterbreak}{\char`\&}}% 
            \def\PYGZlt{\discretionary{}{\Wrappedafterbreak\char`\<}{\char`\<}}% 
            \def\PYGZgt{\discretionary{\char`\>}{\Wrappedafterbreak}{\char`\>}}% 
            \def\PYGZsh{\discretionary{}{\Wrappedafterbreak\char`\#}{\char`\#}}% 
            \def\PYGZpc{\discretionary{}{\Wrappedafterbreak\char`\%}{\char`\%}}% 
            \def\PYGZdl{\discretionary{}{\Wrappedafterbreak\char`\$}{\char`\$}}% 
            \def\PYGZhy{\discretionary{\char`\-}{\Wrappedafterbreak}{\char`\-}}% 
            \def\PYGZsq{\discretionary{}{\Wrappedafterbreak\textquotesingle}{\textquotesingle}}% 
            \def\PYGZdq{\discretionary{}{\Wrappedafterbreak\char`\"}{\char`\"}}% 
            \def\PYGZti{\discretionary{\char`\~}{\Wrappedafterbreak}{\char`\~}}% 
        } 
        % Some characters . , ; ? ! / are not pygmentized. 
        % This macro makes them "active" and they will insert potential linebreaks 
        \newcommand*\Wrappedbreaksatpunct {% 
            \lccode`\~`\.\lowercase{\def~}{\discretionary{\hbox{\char`\.}}{\Wrappedafterbreak}{\hbox{\char`\.}}}% 
            \lccode`\~`\,\lowercase{\def~}{\discretionary{\hbox{\char`\,}}{\Wrappedafterbreak}{\hbox{\char`\,}}}% 
            \lccode`\~`\;\lowercase{\def~}{\discretionary{\hbox{\char`\;}}{\Wrappedafterbreak}{\hbox{\char`\;}}}% 
            \lccode`\~`\:\lowercase{\def~}{\discretionary{\hbox{\char`\:}}{\Wrappedafterbreak}{\hbox{\char`\:}}}% 
            \lccode`\~`\?\lowercase{\def~}{\discretionary{\hbox{\char`\?}}{\Wrappedafterbreak}{\hbox{\char`\?}}}% 
            \lccode`\~`\!\lowercase{\def~}{\discretionary{\hbox{\char`\!}}{\Wrappedafterbreak}{\hbox{\char`\!}}}% 
            \lccode`\~`\/\lowercase{\def~}{\discretionary{\hbox{\char`\/}}{\Wrappedafterbreak}{\hbox{\char`\/}}}% 
            \catcode`\.\active
            \catcode`\,\active 
            \catcode`\;\active
            \catcode`\:\active
            \catcode`\?\active
            \catcode`\!\active
            \catcode`\/\active 
            \lccode`\~`\~ 	
        }
    \makeatother

    \let\OriginalVerbatim=\Verbatim
    \makeatletter
    \renewcommand{\Verbatim}[1][1]{%
        %\parskip\z@skip
        \sbox\Wrappedcontinuationbox {\Wrappedcontinuationsymbol}%
        \sbox\Wrappedvisiblespacebox {\FV@SetupFont\Wrappedvisiblespace}%
        \def\FancyVerbFormatLine ##1{\hsize\linewidth
            \vtop{\raggedright\hyphenpenalty\z@\exhyphenpenalty\z@
                \doublehyphendemerits\z@\finalhyphendemerits\z@
                \strut ##1\strut}%
        }%
        % If the linebreak is at a space, the latter will be displayed as visible
        % space at end of first line, and a continuation symbol starts next line.
        % Stretch/shrink are however usually zero for typewriter font.
        \def\FV@Space {%
            \nobreak\hskip\z@ plus\fontdimen3\font minus\fontdimen4\font
            \discretionary{\copy\Wrappedvisiblespacebox}{\Wrappedafterbreak}
            {\kern\fontdimen2\font}%
        }%
        
        % Allow breaks at special characters using \PYG... macros.
        \Wrappedbreaksatspecials
        % Breaks at punctuation characters . , ; ? ! and / need catcode=\active 	
        \OriginalVerbatim[#1,codes*=\Wrappedbreaksatpunct]%
    }
    \makeatother

    % Exact colors from NB
    \definecolor{incolor}{HTML}{303F9F}
    \definecolor{outcolor}{HTML}{D84315}
    \definecolor{cellborder}{HTML}{CFCFCF}
    \definecolor{cellbackground}{HTML}{F7F7F7}
    
    % prompt
    \makeatletter
    \newcommand{\boxspacing}{\kern\kvtcb@left@rule\kern\kvtcb@boxsep}
    \makeatother
    \newcommand{\prompt}[4]{
        \ttfamily\llap{{\color{#2}[#3]:\hspace{3pt}#4}}\vspace{-\baselineskip}
    }
    

    
    % Prevent overflowing lines due to hard-to-break entities
    \sloppy 
    % Setup hyperref package
    \hypersetup{
      breaklinks=true,  % so long urls are correctly broken across lines
      colorlinks=true,
      urlcolor=urlcolor,
      linkcolor=linkcolor,
      citecolor=citecolor,
      }
    % Slightly bigger margins than the latex defaults
    
    \geometry{verbose,tmargin=1in,bmargin=1in,lmargin=1in,rmargin=1in}
    
    

\begin{document}
    
    \maketitle
    
    

    
    Коцарев/Трајковиќ/Стиков

\hypertarget{ux43cux435ux442ux43eux434ux43eux43bux43eux433ux438ux458ux430-ux43dux430-ux438ux441ux442ux440ux430ux436ux443ux432ux430ux45aux435ux442ux43e-ux432ux43e-ux438ux43aux442}{%
\section{МЕТОДОЛОГИЈА НА ИСТРАЖУВАЊЕТО ВО
ИКТ:}\label{ux43cux435ux442ux43eux434ux43eux43bux43eux433ux438ux458ux430-ux43dux430-ux438ux441ux442ux440ux430ux436ux443ux432ux430ux45aux435ux442ux43e-ux432ux43e-ux438ux43aux442}}

\hypertarget{ux43aux43eux43bux43eux43aux432ux438ux443ux43c-1-5-ux434ux435ux43aux435ux43cux432ux440ux438-2020}{%
\subsection{КОЛОКВИУМ 1 5 декември,
2020}\label{ux43aux43eux43bux43eux43aux432ux438ux443ux43c-1-5-ux434ux435ux43aux435ux43cux432ux440ux438-2020}}

\hypertarget{ux438ux43cux435-ux438-ux43fux440ux435ux437ux438ux43cux435-ux434ux438ux43cux438ux442ux430ux440-ux43cux438ux43bux435ux441ux43aux438-ux431ux440ux43eux458-ux43dux430-ux438ux43dux434ux435ux43aux441-171166}{%
\subparagraph{ИМЕ И ПРЕЗИМЕ: Димитар Милески БРОЈ НА ИНДЕКС:
171166}\label{ux438ux43cux435-ux438-ux43fux440ux435ux437ux438ux43cux435-ux434ux438ux43cux438ux442ux430ux440-ux43cux438ux43bux435ux441ux43aux438-ux431ux440ux43eux458-ux43dux430-ux438ux43dux434ux435ux43aux441-171166}}

    \textbf{1. (15 поени)} За ова прашање ќе треба да најдете оригинален
истражувачки труд на сајтот: Scholar.google.com\\
Трудот треба да има секција за методи (најчесто поднаслов Methods или
Methodology) и да има јасна хипотеза. Бидејќи голем дел од трудовите се
достапни само со плаќање (paywalled), на час ви кажавме како да
пристапите до нив бесплатно. Целиот колоквиум е поврзан со истиот труд,
така што посветете доволно време во изборот на трудот за да можете
полесно да ги одговорите сите прашања и задачи. На час не ви кажавме
како да цитирате труд, така што ова ќе треба сами да го дознаете.
Цитирајте го избраниот труд користејќи го IEEE стилот на цитирање!\\
ОДГОВОР: \cite{durall2019unmasking}

    bibliography.bib

    \begin{tcolorbox}[breakable, size=fbox, boxrule=1pt, pad at break*=1mm,colback=cellbackground, colframe=cellborder]
\prompt{In}{incolor}{3}{\boxspacing}
\begin{Verbatim}[commandchars=\\\{\}]
\PY{c+c1}{\PYZsh{} @article\PYZob{}durall2019unmasking,}
\PY{c+c1}{\PYZsh{}   title=\PYZob{}Unmasking deepfakes with simple features\PYZcb{},}
\PY{c+c1}{\PYZsh{}   author=\PYZob{}Durall, Ricard and Keuper, Margret and Pfreundt, Franz\PYZhy{}Josef and Keuper, Janis\PYZcb{},}
\PY{c+c1}{\PYZsh{}   journal=\PYZob{}arXiv preprint arXiv:1911.00686\PYZcb{},}
\PY{c+c1}{\PYZsh{}   year=\PYZob{}2019\PYZcb{}}
\PY{c+c1}{\PYZsh{} \PYZcb{}}
\end{Verbatim}
\end{tcolorbox}

    paper.tex

    \begin{tcolorbox}[breakable, size=fbox, boxrule=1pt, pad at break*=1mm,colback=cellbackground, colframe=cellborder]
\prompt{In}{incolor}{10}{\boxspacing}
\begin{Verbatim}[commandchars=\\\{\}]
\PY{c+c1}{\PYZsh{} \PYZbs{}begin\PYZob{}abstract\PYZcb{}}
\PY{c+c1}{\PYZsh{} Reproducing paper about DeepFakes with simple Features\PYZbs{}cite\PYZob{}durall2019unmasking\PYZcb{} }
\PY{c+c1}{\PYZsh{} \PYZbs{}end\PYZob{}abstract\PYZcb{}}

\PY{c+c1}{\PYZsh{} \PYZbs{}bibliographystyle\PYZob{}ieee\PYZcb{}}
\PY{c+c1}{\PYZsh{} \PYZbs{}bibliography\PYZob{}bibliography.bib\PYZcb{}}
\end{Verbatim}
\end{tcolorbox}

    Reference\_example.pdf

    \begin{figure}
\centering
\includegraphics{reference.png}
\caption{Reference example}
\end{figure}

    Тука е прикажан начинот за референцирање со помош на .bib датотека и
.tex LaTeX датотека. Во мојот случај јас користев OverLeaf -
https://www.overleaf.com .\\
Исто така има начин за референцирање во Jupyter тетратката која може
отпосле да генерира pdf со референците. За вториот начин е потребни се
ref.bib, citations.tplx, i Compile Article.ipynb кои се наоѓаат во
тековниот директориум.

    \begin{tcolorbox}[breakable, size=fbox, boxrule=1pt, pad at break*=1mm,colback=cellbackground, colframe=cellborder]
\prompt{In}{incolor}{ }{\boxspacing}
\begin{Verbatim}[commandchars=\\\{\}]
\PY{o}{*}\PY{o}{*}\PY{l+m+mf}{2.} \PY{p}{(}\PY{l+m+mi}{45} \PY{n}{поени}\PY{p}{)}\PY{o}{*}\PY{o}{*} \PY{n}{Опишете} \PY{n}{ја} \PY{n}{методологијата} \PY{n}{на} \PY{n}{трудот} \PY{n}{од} \PY{n}{претходното} \PY{n}{прашање} \PY{n}{во} \PY{n}{следните}
\PY{n}{категории}\PY{p}{:}        
\PY{o}{*}\PY{o}{*}\PY{n}{а}\PY{p}{)} \PY{n}{Дали} \PY{n}{истражувањето} \PY{n}{е} \PY{n}{квалитативно} \PY{n}{или} \PY{n}{квантитативно}\PY{err}{?}\PY{o}{*}\PY{o}{*}   
\PY{n}{Конкретното} \PY{n}{истражување} \PY{n}{е} \PY{n}{квантитативно}\PY{o}{.} \PY{n}{Има} \PY{n}{податоци} \PY{n}{кои} \PY{n}{се} \PY{n}{обработуваат} \PY{n}{и} \PY{n}{на} \PY{n}{крајот} \PY{n}{се} \PY{n}{добиваат} \PY{n}{соодветни} \PY{n}{бројчани} \PY{n}{вредности} \PY{n}{со} \PY{n}{кои} \PY{n}{што} \PY{n}{можеме} \PY{n}{нешто} \PY{n}{да} \PY{n}{закчичиме}\PY{o}{.}   
\PY{o}{*}\PY{o}{*}\PY{n}{б}\PY{p}{)} \PY{n}{Како} \PY{n}{се} \PY{n}{собирани} \PY{n}{податоците}\PY{err}{?}\PY{o}{*}\PY{o}{*}     

\PY{o}{*}\PY{o}{*}\PY{n}{в}\PY{p}{)} \PY{n}{Која} \PY{n}{е} \PY{n}{хипотезата} \PY{n}{што} \PY{n}{трудот} \PY{n}{ја} \PY{n}{тестира}\PY{err}{?}\PY{o}{*}\PY{o}{*}   
\PY{o}{*}\PY{o}{*}\PY{n}{г}\PY{p}{)} \PY{n}{Кој} \PY{n}{статистички} \PY{n}{тест} \PY{n}{е} \PY{n}{критериум} \PY{n}{за} \PY{n}{прифаќање}\PY{o}{/}\PY{n}{одбивање} \PY{n}{на} \PY{n}{хипотезата}\PY{err}{?}\PY{o}{*}\PY{o}{*}   
\PY{o}{*}\PY{o}{*}\PY{n}{д}\PY{p}{)} \PY{n}{Какви} \PY{n}{видови} \PY{n}{на} \PY{n}{визуелизација} \PY{n}{се} \PY{n}{користени} \PY{n}{во} \PY{n}{трудот}\PY{err}{?}\PY{o}{*}\PY{o}{*}   
\PY{o}{*}\PY{o}{*}\PY{n}{ѓ}\PY{p}{)} \PY{n}{Дали} \PY{n}{е} \PY{n}{хипотезата} \PY{n}{од} \PY{n}{трудот} \PY{n}{потврдена} \PY{n}{или} \PY{n}{одбиена}\PY{err}{?}\PY{o}{*}\PY{o}{*}          
\end{Verbatim}
\end{tcolorbox}

    \begin{tcolorbox}[breakable, size=fbox, boxrule=1pt, pad at break*=1mm,colback=cellbackground, colframe=cellborder]
\prompt{In}{incolor}{ }{\boxspacing}
\begin{Verbatim}[commandchars=\\\{\}]
    
\PY{l+m+mf}{3.} \PY{p}{(}\PY{l+m+mi}{65} \PY{n}{поени}\PY{p}{)} \PY{n}{Направете} \PY{n}{Jupyter} \PY{n}{тетратката} \PY{n}{поврзана} \PY{n}{со} \PY{n}{трудот} \PY{n}{од} \PY{n}{првото} \PY{n}{прашање} \PY{n}{и}
\PY{n}{прикачете} \PY{n}{ја} \PY{n}{на} \PY{n}{GitHub} \PY{p}{(}\PY{n}{доколку} \PY{n}{немате} \PY{n}{профил} \PY{n}{креирајте} \PY{n}{го}\PY{p}{,} \PY{n}{ќе} \PY{n}{ви} \PY{n}{треба}\PY{p}{)}\PY{o}{.} \PY{n}{Линкот} \PY{n}{од}
\PY{n}{вашиот} \PY{n}{Github} \PY{n}{repo} \PY{n}{мора} \PY{n}{да} \PY{n}{биде} \PY{n}{испратен} \PY{n}{до} \PY{l+m+mf}{23.59} \PY{n}{часот} \PY{n}{на} \PY{l+m+mi}{5} \PY{n}{декември} \PY{p}{(}\PY{n}{сите} \PY{n}{промени} \PY{n}{по}
\PY{n}{овој} \PY{n}{краен} \PY{n}{рок} \PY{n}{нема} \PY{n}{да} \PY{n}{бидат} \PY{n}{прифатени}\PY{p}{)}\PY{o}{.} \PY{n}{Исто} \PY{n}{така} \PY{n}{нема} \PY{n}{да} \PY{n}{прифаќаме} \PY{n}{тетратки}
\PY{n}{хостирани} \PY{n}{на} \PY{n}{било} \PY{n}{кое} \PY{n}{друго} \PY{n}{место} \PY{n}{освен} \PY{n}{на} \PY{n}{Github}\PY{o}{.}
\PY{n}{а}\PY{p}{)} \PY{n}{Тетратката} \PY{n}{треба} \PY{n}{да} \PY{n}{започне} \PY{n}{со} \PY{n}{краток} \PY{n}{опис} \PY{n}{на} \PY{n}{трудот} \PY{p}{(}\PY{n}{напишан} \PY{n}{во} \PY{n}{Markdown}\PY{p}{)}\PY{o}{.}
\PY{n}{Краткиот} \PY{n}{опис} \PY{n}{треба} \PY{n}{во} \PY{n}{стотина} \PY{n}{зборови} \PY{n}{да} \PY{n}{објасни} \PY{n}{зошто} \PY{n}{е} \PY{n}{овој} \PY{n}{труд} \PY{n}{значаен}\PY{o}{.}
\PY{n}{б}\PY{p}{)} \PY{n}{Остатокот} \PY{n}{од} \PY{n}{тетратката} \PY{n}{го} \PY{n}{оставаме} \PY{n}{на} \PY{n}{вас}\PY{o}{.} \PY{n}{Не} \PY{n}{заборавајте} \PY{n}{дека} \PY{n}{колоквиумите}
\PY{n}{ќе} \PY{n}{бидат} \PY{n}{рангирани}\PY{p}{,} \PY{n}{така} \PY{n}{што} \PY{n}{тие} \PY{n}{кои} \PY{n}{ќе} \PY{n}{имаат} \PY{n}{најквалитетна} \PY{n}{тетратка} \PY{n}{ќе} \PY{n}{добијат} \PY{n}{најмногу}
\PY{n}{поени}\PY{o}{.} \PY{n}{За} \PY{n}{да} \PY{n}{биде} \PY{n}{кандидат} \PY{n}{за} \PY{n}{максимална} \PY{n}{оценка}\PY{p}{,} \PY{n}{тетратката} \PY{n}{треба} \PY{n}{да} \PY{n}{содржи} \PY{n}{три} \PY{n}{од}
\PY{n}{овие} \PY{l+m+mi}{5} \PY{n}{карактеристки}\PY{p}{:}
\PY{o}{\PYZhy{}} \PY{n}{Формули} \PY{n}{од} \PY{n}{избраниот} \PY{n}{труд} \PY{n}{напишани} \PY{n}{во} \PY{n}{LaTeX}
\PY{o}{\PYZhy{}} \PY{n}{Ќелии} \PY{n}{со} \PY{n}{код} \PY{n}{од} \PY{n}{избраниот} \PY{n}{труд} \PY{n}{кои} \PY{n}{може} \PY{n}{да} \PY{n}{се} \PY{n}{егзекутираат} \PY{p}{(}\PY{n}{полесно} \PY{n}{е} \PY{n}{ова}
\PY{n}{да} \PY{n}{се} \PY{n}{направи} \PY{n}{доколку} \PY{n}{податоците} \PY{n}{и} \PY{n}{кодот} \PY{n}{од} \PY{n}{трудот} \PY{n}{се} \PY{n}{јавно} \PY{n}{достапни}\PY{p}{)}
\PY{o}{\PYZhy{}} \PY{n}{Интерактивна} \PY{n}{визуелизација} \PY{p}{(}\PY{n}{Plotly}\PY{p}{,} \PY{n}{ipywidgets} \PY{n}{или} \PY{n}{други} \PY{n}{алатки}\PY{p}{)}
\PY{o}{\PYZhy{}} \PY{n}{Вметнатно} \PY{n}{лого} \PY{n}{на} \PY{n}{журналот} \PY{n}{во} \PY{n}{кој} \PY{n}{е} \PY{n}{објавен} \PY{n}{трудот}
\PY{o}{\PYZhy{}} \PY{n}{Ембедиран} \PY{n}{мултимедијален} \PY{n}{запис} \PY{n}{поврзан} \PY{n}{со} \PY{n}{трудот} \PY{p}{(}\PY{n}{YouTube} \PY{n}{видео}\PY{p}{,}
\PY{n}{podcast}\PY{p}{,} \PY{o}{.}\PY{o}{.}\PY{o}{.}\PY{p}{)}
\end{Verbatim}
\end{tcolorbox}

    Целта на ова прашање е да бидете креативни. Понудете ни тетратка која го
надополнува оригиналниот PDF и го прави истражувањето да биде покорисно.
Доколку трудот ги споделува податоците, тогаш можете да направите и
сосема нова визуелизација. Изненадете нè! P.S. Вашитe одговори на
колоквиумот треба да бидат прикачени на GitHub (во PDF или друг
електронски формат) заедно со Jupyter тетратката.

    \begin{tcolorbox}[breakable, size=fbox, boxrule=1pt, pad at break*=1mm,colback=cellbackground, colframe=cellborder]
\prompt{In}{incolor}{ }{\boxspacing}
\begin{Verbatim}[commandchars=\\\{\}]

\end{Verbatim}
\end{tcolorbox}


    % Add a bibliography block to the postdoc
    
    
\bibliographystyle{ieee}
\bibliography{ref}

    
\end{document}
